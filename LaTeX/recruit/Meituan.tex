\documentclass[a4paper]{ctexart}

\usepackage{tabularx} % extra features for tabular environment
\usepackage{amsmath}  % improve math presentation
\usepackage{graphicx} % takes care of graphic including machinery
\usepackage[margin=1in,letterpaper]{geometry} % decreases margins
\usepackage{cite} % takes care of citations
\usepackage[final]{hyperref} % adds hyper links inside the generated pdf file
\usepackage{ctex}
\usepackage{titlesec}
%\usepackage{CJKutf8, CJK}
\usepackage{makecell}                 % 三线表-竖线
\usepackage{booktabs}                 % 三线表-短细横线
% \usepackage{natbib}
\usepackage{graphicx}				  % 表格单元格逆时针
\usepackage{multirow}				  % 合并单元格
\usepackage{array}
\usepackage{amssymb}				  % 勾
\usepackage{amsmath}
\usepackage{longtable}                % 导入 longtable 宏包,表格自动换行
\usepackage{caption}
\usepackage{subcaption}               % 设置子图
\usepackage{caption}
%\usepackage{subfigure}
\usepackage{diagbox}
\usepackage{color}					  % 文本颜色包
\usepackage{colortbl}
\usepackage{xcolor}
\usepackage{bbm}					  % 输入指示函数
\usepackage{tablefootnote}			  % 表格注释
\usepackage{pythonhighlight}
%\usepackage{fancyhdr}
\usepackage{lastpage}
\usepackage{tocloft}
\usepackage{authblk}
\usepackage{setspace}
\usepackage{float}
\usepackage[section]{placeins}

% 设置页面边距
\geometry{a4paper, top=1.7cm, bottom=1.6cm, left=1.6cm, right=1.6cm}

%\pagestyle{fancy}
%\fancyhf{}
%\fancyhead{}
%\fancyfoot{}
%\fancyhead[R]{\small Page \thepage\ of \pageref*{LastPage}}
%\fancyhead[L]{\zihao{-5} \songti 开题报告}

\usepackage{listings}                 % 导入代码块
\usepackage{xcolor}

% 定义适合黑色背景的配色方案
\definecolor{vscodeblue}{RGB}{97, 175, 239}  % keywords
\definecolor{vscodegreen}{RGB}{152, 195, 121}% comments
\definecolor{vscodepurple}{RGB}{198, 120, 221}% strings
\definecolor{vscodegray}{RGB}{153,153,153}   % line numbers
\definecolor{vscodeorange}{RGB}{224,108,117} % function names
\definecolor{vscodebackground}{RGB}{40,44,52} % 黑色背景
\definecolor{vscodewhite}{RGB}{248,248,242}  % 白色字体

% 设置 C++ 代码块的样式,适合黑色背景
\lstset{
	language=java,
	backgroundcolor=\color{vscodebackground},  % 黑色背景
	basicstyle=\ttfamily\footnotesize\color{vscodewhite}, % 白色文字,紧凑的字体大小
	keywordstyle=\color{vscodeblue}\bfseries,  % 关键字颜色
	commentstyle=\color{vscodegreen}\itshape,  % 注释颜色
	stringstyle=\color{vscodepurple},          % 字符串颜色
	numberstyle=\tiny\color{vscodegray},       % 行号颜色
	numbers=left,                              % 行号显示在左侧
	stepnumber=1,                              % 每行都显示行号
	numbersep=5pt,                             % 行号与代码之间的距离
	tabsize=4,                                 % tab 键宽度
	showspaces=false,                          % 不显示空格符号
	showstringspaces=false,                    % 字符串中的空格不显示特殊符号
	breaklines=true,                           % 自动换行
	breakatwhitespace=true,                    % 只在空格处换行
	columns=fullflexible,                      % 紧凑的代码对齐
	keepspaces=true,                           % 保持空格符号
	frame=single,                              % 给代码块加框
	framesep=3pt,                              % 代码与框之间的距离
	rulecolor=\color{vscodegray},              % 框的颜色
	escapeinside={(*@}{@*)},                   % 允许LaTeX注释
	xleftmargin=10pt,                          % 左边距
	xrightmargin=10pt,                         % 右边距
}

\hypersetup{
	colorlinks=true,       % false: boxed links; true: colored links
	linkcolor=blue,        % color of internal links
	citecolor=blue,        % color of links to bibliography
	filecolor=magenta,     % color of file links
	urlcolor=blue         
}
%++++++++++++++++++++++++++++++++++++++++
\titleformat{\section}{\Large\bfseries}{\thesection}{1em}{}
\titleformat{\subsection}{\large\bfseries}{\thesubsection}{1em}{}
\titleformat{\subsubsection}{\normalsize\bfseries}{\thesubsubsection}{1em}{}
\titleformat{\paragraph}[runin]{\normalsize\bfseries}{\paragraph}{1em}{}
\titleformat{\subparagraph}[runin]{\normalsize\bfseries}{\subparagraph}{1em}{}

\begin{document}
	
\section{订单编号格式校验}
	
\begin{center}
	\Large \textbf{订单编号格式校验}
\end{center}
	
\vspace{10pt}
	
\noindent\textbf{题目描述:}
	
在一个在线购物系统中,每个订单都有一个唯一的订单编号。订单编号由三部分组成:前缀、日期部分和序列号部分。前缀由大写字母组成,日期部分由 8 位数字组成,表示 \texttt{YYYYMMDD} 格式的日期,序列号部分由 4 位数字组成。例如:\texttt{"ORD202306151234"} 现在需要你编写一个程序来判断这些订单编号是否符合以下规则:
\begin{itemize}
	\item[1.] 前缀部分必须是大写字母,长度为3;
	\item[2.] 日期部分必须是 8 位数字,且是一个有效的日期\texttt{(YYYYMMDD)}。
	\item[3.] 序列号部分必须是 4 位数字。
	\item[4.] 编号必须以字母部分开头,日期部分中间,序列号部分结尾。
\end{itemize}
	
如果符合规则,输出\texttt{"valid"};
	
如果不符合规则,输出\texttt{"invalid"}。
	
提示:日期这里需要判断闰年。闰年判断的条件是能被 4 整除,但不能被 100 整除;或能被 400 整除。
	
\noindent\textbf{输入描述:}
	
第一行包含数据组数 n。
	
接下来的 n 行,每一行包含一个需要判断的订单编号字符串。
	
\noindent\textbf{输出描述:} 
	
对于每一行,输出判断的结果。
	
\noindent\textbf{样例输入1:}
	
\lstset{numbers=none}
\begin{lstlisting}
6
ORD202306151234
ORD202313151234
ORD20230615123A
ORD20230615123
ORD2023061512345
ORD202306151234
\end{lstlisting}
\lstset{numbers=left}
	
\noindent\textbf{样例输出1:}
\lstset{numbers=none}
\begin{lstlisting}
valid
invalid
invalid
invalid
invalid
valid
\end{lstlisting}
\lstset{numbers=left}
	

	
\vspace{10pt}
	
\noindent\textbf{思路分析:}
\begin{itemize}
	\item[1.] 读取输入:
	\begin{itemize}
		\item[·] 使用 Scanner 类读取输入,首先获取输入的行数 n,然后依次读取每一行的订单编号。
	\end{itemize}

	\item[2.] 验证订单编号:
	\begin{itemize}
		\item[·] 对于每个订单编号,首先检查其长度是否为 15。如果长度不为 15,则直接返回 "invalid"。
		
		\item[·] 如果长度为 15,则继续从字符串中提取订单编号的三部分:前缀、日期和序列号。
		
		\item[·] 分别对这三部分进行验证,确保其格式正确。
		
		\item[·] 如果所有部分都合法,则返回 "valid",否则返回 "invalid"。
	\end{itemize}

	\item[3.] 验证逻辑:
	\begin{itemize}
		\item[·] 前缀验证:提取订单编号的前3个字符,使用正则表达式 [A-Z]{3} 来确保前缀是3个大写字母。如果不匹配,返回 "invalid"。
	
		\item[·] 日期验证:提取订单编号的第4到第11个字符,使用正则表达式 \\d{8} 确保日期部分是8位数字。如果不匹配或者日期无效,则返回 "invalid"。具体的日期验证通过检查年、月、日是否合法。
	
		\item[·] 序列号验证:提取订单编号的最后4个字符,使用正则表达式 \\d{4} 确保序列号部分为4位数字。如果不匹配,返回 "invalid"。
	\end{itemize}
	
	\item[4.] 日期合法性检查:
	\begin{itemize}
		\item[·] 年份:提取日期的前4位作为年份,判断是否为闰年(闰年的条件是能被4整除且不能被100整除,或者能被400整除)。
	
		\item[·] 月份:提取日期的中间两位作为月份,判断月份是否在1到12之间。
	
		\item[·] 日期:提取日期的最后两位作为日,检查这个日是否符合该月份的天数。比如2月最多28天,但如果是闰年则为29天。
	\end{itemize}
	
	\item[5.] 判断闰年:
	\begin{itemize}
		\item[·] 使用 \texttt{isLeapYear} 函数来判断一个年份是否是闰年。闰年的规则为:如果年份能被 4 整除但不能被 100 整除,或者能被 400 整除,则该年为闰年。
	\end{itemize}
	
\end{itemize}	
	
\noindent\textbf{实现代码:}
	
\begin{lstlisting}
// 主类,入口方法
public class Main {
	public static void main(String[] args) {
		// 使用Scanner从控制台获取输入
		Scanner sc = new Scanner(System.in);
		int n = sc.nextInt();  // 读取订单编号的数量
		sc.nextLine();  // 消耗换行符
		
		// 循环处理每一个订单编号
		for (int i = 0; i < n; i++) {
			String code = sc.nextLine();  // 读取一行订单编号
			System.out.println(validate(code));  // 验证该订单编号是否合法并输出结果
		}
		
		sc.close();  // 关闭Scanner
	}
	
	// 验证订单编号是否合法的主方法
	private static String validate(String code) {
		// 如果长度不为15,直接返回"invalid"
		if (code.length() != 15) {
			return "invalid";
		}
		
		// 提取订单编号的前缀、日期部分和序列号部分
		String pre = code.substring(0, 3);  // 前缀部分
		String date = code.substring(3, 11);  // 日期部分
		String serial = code.substring(11, 15);  // 序列号部分
		
		// 验证前缀部分是否为三个大写字母
		if (!pre.matches("[A-Z]{3}")) {
			return "invalid";
		}
		
		// 验证日期部分是否合法
		if (!isValidDate(date)) {
			return "invalid";
		}
		
		// 验证序列号部分是否为四位数字
		if (!serial.matches("\\d{4}")) {
			return "invalid";
		}
		
		// 如果所有部分都合法,返回"valid"
		return "valid";
	}
	
	// 验证日期是否合法的辅助方法
	private static boolean isValidDate(String date) {
		// 如果日期部分长度不是8位,或者不全是数字,返回false
		if (date.length() != 8 || !date.matches("\\d{8}")) {
			return false;
		}
		
		// 提取年份、月份和日期
		int y = Integer.parseInt(date.substring(0, 4));  // 年份
		int m = Integer.parseInt(date.substring(4, 6));  // 月份
		int d = Integer.parseInt(date.substring(6, 8));  // 日期
		
		// 检查月份是否在1到12之间
		if (m < 1 || m > 12) {
			return false;
		}
		
		// 检查日期是否大于0
		if (d < 1) {
			return false;
		}
		
		// 每个月的最大天数
		int[] days = {31, 28, 31, 30, 31, 30, 31, 31, 30, 31, 30, 31};
		// 如果是闰年,2月的天数应为29天
		if (isLeapYear(y)) {
			days[1] = 29;
		}
		
		// 判断日期是否在该月的最大天数之内
		return d <= days[m - 1];
	}
	
	// 判断是否为闰年的辅助方法
	private static boolean isLeapYear(int y) {
		// 闰年的条件:能被4整除且不能被100整除,或者能被400整除
		return (y % 4 == 0 && y % 100 != 0) || (y % 400 == 0);
	}
}

\end{lstlisting}


\newpage

\section{统计以大写字母开头的单词个数}

\begin{center}
	\Large \textbf{统计以大写字母开头的单词个数}
\end{center}

\vspace{10pt}

\noindent\textbf{题目描述:}

小美写单词喜欢横着写,她记录了若干个人的名字,但是不小心加进去了一些无关的单词。

一个名字单词以大写字母开头,请你帮助她统计共有多少个人的名字。

\noindent\textbf{输入描述:}

在一行上输入一个长度为 $n (1\leq n \leq 10^5)$、且由大小写字母和空格混合构成的字符串 $s$ 代表小美的全部单词,每个单词之间使用空格间隔。

除此之外,保证字符串的开头与结尾字符不为空格。

\noindent\textbf{输出描述:} 

在一行上输出一个整数,代表人名的个数。

\noindent\textbf{样例输入1:}
\lstset{numbers=none}
\begin{lstlisting}
ABC abc Abc
\end{lstlisting}
\lstset{numbers=left}

\noindent\textbf{样例输出1:}
\lstset{numbers=none}
\begin{lstlisting}
2	
\end{lstlisting}
\lstset{numbers=left}

\noindent\textbf{样例输入2:}
\lstset{numbers=none}
\begin{lstlisting}
A A c
\end{lstlisting}
\lstset{numbers=left}

\noindent\textbf{样例输出2:}
\lstset{numbers=none}
\begin{lstlisting}
2	
\end{lstlisting}
\lstset{numbers=left}

\vspace{10pt}

\noindent\textbf{思路分析:}
\begin{itemize}
	\item[1.] 字符串拆分:
	
	我们需要把输入字符串按空格拆分成若干个单词,这里可以使用 Java 语言自带的 \texttt{split} 函数,这个函数会把字符串根据空格分隔成一个单词数组。
	
	\item[2.] 判断条件:
	
	\begin{itemize}
	\item[·] 遍历每个单词,检查它的首字符是否是大写字母。可以通过调用\texttt{Character.isUpperCase()}(Java) 或者直接比较字符的 \texttt{ASCII} 值来判断。
	
	\item[·] 如果某个单词的首字符是大写字母,则这个单词计数为一个名字。
	\end{itemize}
	
	\item[3.] 输出结果:最后将计数器的值输出,表示以大写字母开头的单词数量。
\end{itemize}

\noindent\textbf{实现代码:}

\begin{lstlisting}
import java.util.Scanner;

public class Main {
	public static void main(String[] args) {
		// 创建Scanner对象用于读取用户输入
		Scanner sc = new Scanner(System.in);
		
		// 读取一整行的输入,作为字符串存入变量s
		String s = sc.nextLine();
		
		// 将字符串按空格拆分成单词数组,存入数组w
		String[] w = s.split(" ");
		
		// 初始化计数器,记录以大写字母开头的单词个数
		int cnt = 0;
		
		// 遍历数组中的每一个单词
		for (String x : w) {
			// 如果单词的首字母是大写字母,计数器加1
			if (Character.isUpperCase(x.charAt(0))) {
				cnt++;
			}
		}
		
		// 输出计数器的最终值,即以大写字母开头的单词个数
		System.out.println(cnt);
	}
}

\end{lstlisting}
	
\newpage
	
\section{最小种树区间长度}
	
\begin{center}
	\Large \textbf{最小种树区间长度}
\end{center}
	
\vspace{10pt}
	
\noindent\textbf{题目描述:}

长度无限长的公路上,小美雇佣了 $n$ 位工人来种树,每个点最多种一棵树。
	
从左向右数,工人所站的位置为 $a_1, a_2, \cdots, a_n$。已知每位工人都会将自己所在位置的右侧一段长度的区间种满树,且每位工人的种树区间长度相同。
	
现在小美希望公路上至少有 $k$ 棵树,为了节约成本,他希望每位工人种树的区间长度尽可能短,请你帮他求出,工人们的种树区间至少多长,才能使得公路被种上至少 $k$ 棵树。
	
\noindent\textbf{输入描述:}
	
第一行输入两个正整数 $n, k(1 \leq n, k \leq 2 \times 10^5)$,分别表示工人的数量,以及小美要求树的最少数量。
	
第二行输入 $n$个正整数 $a_1, a_2, \cdots, a_n (1 \leq a_i \leq 2 \times 10^5)$,表示每名工人的位置。
	
\noindent\textbf{输出描述:} 
	
在一行上输出一个整数,代表工人们最短的种树区间长度。
	
\noindent\textbf{样例输入1:}
	
\lstset{numbers=none}
\begin{lstlisting}
3 6
1 2 5
\end{lstlisting}
\lstset{numbers=left}
	
\noindent\textbf{样例输出1:}
\lstset{numbers=none}
\begin{lstlisting}
3	
\end{lstlisting}
\lstset{numbers=left}
	
\noindent\textbf{说明:}

每位工人种树的区间长度至少为 3.

这样以来:

第一名工人种:\texttt{1,2,3} 点的树。

第二名工人种:\texttt{2,3,4} 点的树。

第三名工人种:\texttt{5,6,7} 点的树。

由于每个位置最多种一棵树,因此共有:\texttt{1, 2, 3, 4, 5, 6, 7} 这些点有树,满足至少 $k = 6$ 棵树。

可以证明,不存在比 3 更小的答案,

	
\vspace{10pt}
	
\noindent\textbf{思路分析:}
\begin{itemize}
\item[1.] 排序工人位置:
\begin{itemize}
	\item[·] 由于工人的位置是无序的,为了简化计算,我们首先对工人的位置进行排序。这样,种树的区间就可以方便地计算和遍历。
\end{itemize}

\item[2.] 二分查找确定最小区间长度:
\begin{itemize}
	\item[·] 二分查找的本质是 在一定的区间范围内查找最优解,这里我们需要查找最短的区间长度使得能种下至少 \texttt{k} 棵树。
	\item[·] 定义 \texttt{left} 为最短的可能区间(即1),\texttt{right} 为最大的可能区间(即工人位置的最大值与最小值之差)。
\end{itemize}

\item[3.] 核心函数 \texttt{canPlantTrees}:
\begin{itemize}
	\item[·] 在给定区间长度下,判断是否可以种至少 \texttt{k} 棵树。这个函数的核心是通过遍历工人的位置,计算每个工人能种的树的数量,并统计总的树数。如果总树数达到了 \texttt{k},则返回 \texttt{true}。
\end{itemize}


\end{itemize}
	
\noindent\textbf{实现代码:}
	
\begin{lstlisting}
import java.util.Arrays;
import java.util.Scanner;

public class Main {
	public static void main(String[] args) {
		Scanner sc = new Scanner(System.in);
		
		// 读取输入值,n是工人数,k是需要种的树的数量
		int n = sc.nextInt();
		int k = sc.nextInt();
		int[] pos = new int[n];  // 存储每个工人的位置
		for (int i = 0; i < n; i++) {
			pos[i] = sc.nextInt();
		}
		
		// 对工人位置排序,方便后续计算
		Arrays.sort(pos);
		
		// 二分查找区间长度,left是最小区间长度,right是最大可能区间长度
		int left = 1;
		int right = pos[n - 1] - pos[0] + 1;
		int res = right;  // 保存最终结果
		
		// 开始二分查找
		while (left <= right) {
			int mid = (left + right) / 2;  // 中间的区间长度
			
			// 如果当前区间长度能种够k棵树,则尝试更小的区间长度
			if (canPlant(pos, n, k, mid)) {
				res = mid;
				right = mid - 1;
			} else {
				left = mid + 1;
			}
		}
		
		// 输出最小的区间长度
		System.out.println(res);
	}
	
	// 辅助函数,判断给定区间长度len时,能否种下至少k棵树
	private static boolean canPlant(int[] pos, int n, int k, int len) {
		int trees = 0;  // 已种的树的数量
		int lastPos = -1;  // 上一次种树的位置
		
		for (int i = 0; i < n; i++) {
			int start = pos[i];  // 当前工人的起始位置
			int end = start + len - 1;  // 工人可以种树的终止位置
			
			// 如果上一次种树的位置在当前工人的起始位置之前,直接种树
			if (lastPos < start) {
				trees += len;  // 可以种下整整一个区间长度的树
				lastPos = end;  // 更新最后种树的位置
			}
			// 如果上一次种树的位置在当前工人区域内,则只种剩下的部分
			else if (lastPos < end) {
				trees += end - lastPos;  // 计算剩下的区域长度并种树
				lastPos = end;  // 更新最后种树的位置
			}
			
			// 如果种下的树的数量已经达到或超过k棵,返回true
			if (trees >= k) {
				return true;
			}
		}
		
		// 最后检查是否种了足够的树
		return trees >= k;
	}
}

\end{lstlisting}
	
	
\end{document}
