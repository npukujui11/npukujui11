\documentclass[a4paper]{ctexart}

\usepackage{tabularx} % extra features for tabular environment
\usepackage{amsmath}  % improve math presentation
\usepackage{graphicx} % takes care of graphic including machinery
\usepackage[margin=1in,letterpaper]{geometry} % decreases margins
\usepackage{cite} % takes care of citations
\usepackage[final]{hyperref} % adds hyper links inside the generated pdf file
\usepackage{ctex}
\usepackage{titlesec}
%\usepackage{CJKutf8, CJK}
\usepackage{makecell}                 % 三线表-竖线
\usepackage{booktabs}                 % 三线表-短细横线
% \usepackage{natbib}
\usepackage{graphicx}				  % 表格单元格逆时针
\usepackage{multirow}				  % 合并单元格
\usepackage{array}
\usepackage{amssymb}				  % 勾
\usepackage{amsmath}
\usepackage{longtable}                % 导入 longtable 宏包,表格自动换行
\usepackage{caption}
\usepackage{subcaption}               % 设置子图
\usepackage{caption}
%\usepackage{subfigure}
\usepackage{diagbox}
\usepackage{color}					  % 文本颜色包
\usepackage{colortbl}
\usepackage{xcolor}
\usepackage{bbm}					  % 输入指示函数
\usepackage{tablefootnote}			  % 表格注释
\usepackage{pythonhighlight}
%\usepackage{fancyhdr}
\usepackage{lastpage}
\usepackage{tocloft}
\usepackage{authblk}
\usepackage{setspace}
\usepackage{float}
\usepackage[section]{placeins}

% 设置页面边距
\geometry{a4paper, top=1.7cm, bottom=1.6cm, left=1.6cm, right=1.6cm}

%\pagestyle{fancy}
%\fancyhf{}
%\fancyhead{}
%\fancyfoot{}
%\fancyhead[R]{\small Page \thepage\ of \pageref*{LastPage}}
%\fancyhead[L]{\zihao{-5} \songti 开题报告}

\usepackage{listings}                 % 导入代码块
\usepackage{xcolor}

% 定义适合黑色背景的配色方案
\definecolor{vscodeblue}{RGB}{97, 175, 239}  % keywords
\definecolor{vscodegreen}{RGB}{152, 195, 121}% comments
\definecolor{vscodepurple}{RGB}{198, 120, 221}% strings
\definecolor{vscodegray}{RGB}{153,153,153}   % line numbers
\definecolor{vscodeorange}{RGB}{224,108,117} % function names
\definecolor{vscodebackground}{RGB}{40,44,52} % 黑色背景
\definecolor{vscodewhite}{RGB}{248,248,242}  % 白色字体

% 设置 C++ 代码块的样式,适合黑色背景
\lstset{
	language=C++,
	backgroundcolor=\color{vscodebackground},  % 黑色背景
	basicstyle=\ttfamily\footnotesize\color{vscodewhite}, % 白色文字,紧凑的字体大小
	keywordstyle=\color{vscodeblue}\bfseries,  % 关键字颜色
	commentstyle=\color{vscodegreen}\itshape,  % 注释颜色
	stringstyle=\color{vscodepurple},          % 字符串颜色
	numberstyle=\tiny\color{vscodegray},       % 行号颜色
	numbers=left,                              % 行号显示在左侧
	stepnumber=1,                              % 每行都显示行号
	numbersep=5pt,                             % 行号与代码之间的距离
	tabsize=4,                                 % tab 键宽度
	showspaces=false,                          % 不显示空格符号
	showstringspaces=false,                    % 字符串中的空格不显示特殊符号
	breaklines=true,                           % 自动换行
	breakatwhitespace=true,                    % 只在空格处换行
	columns=fullflexible,                      % 紧凑的代码对齐
	keepspaces=true,                           % 保持空格符号
	frame=single,                              % 给代码块加框
	framesep=3pt,                              % 代码与框之间的距离
	rulecolor=\color{vscodegray},              % 框的颜色
	escapeinside={(*@}{@*)},                   % 允许LaTeX注释
	xleftmargin=10pt,                          % 左边距
	xrightmargin=10pt,                         % 右边距
}

\hypersetup{
	colorlinks=true,       % false: boxed links; true: colored links
	linkcolor=blue,        % color of internal links
	citecolor=blue,        % color of links to bibliography
	filecolor=magenta,     % color of file links
	urlcolor=blue         
}
%++++++++++++++++++++++++++++++++++++++++
\titleformat{\section}{\Large\bfseries}{\thesection}{1em}{}
\titleformat{\subsection}{\large\bfseries}{\thesubsection}{1em}{}
\titleformat{\subsubsection}{\normalsize\bfseries}{\thesubsubsection}{1em}{}
\titleformat{\paragraph}[runin]{\normalsize\bfseries}{\paragraph}{1em}{}
\titleformat{\subparagraph}[runin]{\normalsize\bfseries}{\subparagraph}{1em}{}

\begin{document}

\section{小A烤面包}

\begin{center}
	\Large \textbf{小A烤面包}
\end{center}

\noindent\textbf{时间限制:} 3000MS

\noindent\textbf{内存限制:} 589824KB

\vspace{10pt}

\noindent\textbf{题目描述:}

小 $A$ 每天都要吃 $a, b$两种面包名一个。而他有 $n$ 个不同的面包机,不同面包机制作面包的时间各不相同。第 $i$ 台面包机制作 $a$ 面包需要花费 $a_i$ 的时间,制作 $b$ 面包则需要花费 $b_i$ 的时间。为能尽快吃到这两种面包,小 $A$ 可以选择两个不同的面包机 $x, y$ 同时工作,并分别制作 $a,b$ 两种面包,花费的时间将是 $\max(a_x, b_y)$。当然,小 $A$ 也可以选择其中一个面包机 $x$ 制作 $a,b$ 两种面包,花费的时间将是 $a_x + b_x$。为能尽快吃到面包,请你帮小 $A$ 计算一下,至少需要花费多少时间才能完成这两种面包的制作。

\noindent\textbf{输入描述:}

第一行一个正整数 $n$,表示面包机的个数。

第二行 $n$ 个正整数 $a_i$,表示面包机制作面包 $a$ 的时间。

第三行 $n$ 个正整数 $b_i$,表示面包机制作面包 $b$ 的时间。

$1 \leq n \leq 10^5, 1 \leq a_i, b_i \leq 10^5$

\noindent\textbf{输出描述:} 

输出一行一个正整数,表示需要花费的最少时间。

\noindent\textbf{样例输入1:}

\lstset{numbers=none}
\begin{lstlisting}
3
2 5 9
4 3 6
\end{lstlisting}
\lstset{numbers=left}

\noindent\textbf{样例输出1:}
\lstset{numbers=none}
\begin{lstlisting}
3
\end{lstlisting}
\lstset{numbers=left}

样例1提示:直接将序列中唯一的元素删去即可

\noindent\textbf{样例输入2:}

\lstset{numbers=none}
\begin{lstlisting}
3
2 5 7
2 8 6
\end{lstlisting}
\lstset{numbers=left}

\noindent\textbf{样例输出2:}
\lstset{numbers=none}
\begin{lstlisting}
4
\end{lstlisting}
\lstset{numbers=left}

样例2提示:可能的一种操作为,删去最后一个元素,再使第一个元素加一,得到的序列为2 3。

\vspace{10pt}

\noindent\textbf{思路分析:}




\noindent\textbf{实现代码:}

\begin{lstlisting}
#include <iostream>
#include <vector>
#include <algorithm>
#include <climits>
#include <cstdio>

using namespace std;

int main() {
	int n;
	
	// 使用 scanf 读取输入
	scanf("%d", &n);
	
	vector<int> a(n), b(n);
	
	// 读取 a 数组
	for (int i = 0; i < n; i++) {
		scanf("%d", &a[i]);
	}
	
	// 读取 b 数组
	for (int i = 0; i < n; i++) {
		scanf("%d", &b[i]);
	}
	
	// 初始化最小时间为较大的值
	int min_time_same = INT_MAX;
	int min_time_diff = INT_MAX;
	
	// 情况1:同一台机器制作两种面包
	for (int i = 0; i < n; i++) {
		min_time_same = min(min_time_same, a[i] + b[i]);
	}
	
	// 情况2:两台不同的机器制作面包
	for (int i = 0; i < n; i++) {
		for (int j = 0; j < n; j++) {
			if (i != j) {
				min_time_diff = min(min_time_diff, max(a[i], b[j]));
			}
		}
	}
	
	// 取两种情况的最小值
	int result = min(min_time_same, min_time_diff);
	
	// 输出最小时间
	printf("%d\n", result);
	
	return 0;
}
\end{lstlisting}

\newpage

\section{序列修改}

\begin{center}
	\Large \textbf{序列修改}
\end{center}

\noindent\textbf{时间限制:} 3000MS

\noindent\textbf{内存限制:} 589824KB

\vspace{10pt}

\noindent\textbf{题目描述:}

给一个长度为 $n$ 的序列和一个整数 $x$,每次操作可以选择序列中的一个元素,将其从序列中删去,或者将其值加一。问至少操作多少次,可以使操作后的序列(可以为空)中数字之和是 $x$ 的倍数。

\noindent\textbf{输入描述:}

第一行两个用空格隔开的正整数 $n$ 和 $x$,含义如问题描述中所述。

第二行是 $n$ 个用空格隔开的正整数$A[1], A[2], \cdots, A[n]$,表示序列中 $n$ 个元素的值。

$1 \leq n \leq 1000, 1 \leq x \leq 1000,1 \leq A[i] \leq 1000$

\noindent\textbf{输出描述:} 

一行一个整数,表示使序列中数字之和是 $x$ 的倍数所需要的最少操作数.

\noindent\textbf{样例输入1:}

\lstset{numbers=none}
\begin{lstlisting}
1 3
4
\end{lstlisting}
\lstset{numbers=left}

\noindent\textbf{样例输出1:}
\lstset{numbers=none}
\begin{lstlisting}
1
\end{lstlisting}
\lstset{numbers=left}

样例1提示:直接将序列中唯一的元素删去即可

\noindent\textbf{样例输入2:}

\lstset{numbers=none}
\begin{lstlisting}
3 5
1 3 3
\end{lstlisting}
\lstset{numbers=left}

\noindent\textbf{样例输出2:}
\lstset{numbers=none}
\begin{lstlisting}
2
\end{lstlisting}
\lstset{numbers=left}

样例2提示:可能的一种操作为,删去最后一个元素,再使第一个元素加一,得到的序列为2 3。

\vspace{10pt}

\noindent\textbf{思路分析:}

首先判断出这是一道动态规划的算法题,对于动态规划问题,在考虑复杂度的前提下,一般有以下方法:
\begin{itemize}
	\item[(1)] 基于备忘录的递归方法;
	\item[(2)] 迭代法;
\end{itemize}

我个人一般倾向于使用基于备忘录的递归方法,首先递归方法即是穷举法,只是空间复杂度低了,剩下就是对树的裁剪优化,使用备忘录是一个很好的优化方法,相较于迭代法,严格依赖于状态转移方程的正确性,而递归只需要考虑退出递归的条件和每一步状态转移的操作。

\begin{itemize}
	\item[1.] 问题定义:
	\begin{itemize}
		\item[·] 我们有一个序列 A 和目标倍数 x。我们要通过删除元素或者加一操作,修改序列使其数字之和是 x 的倍数。
		\item[·] 最终需要最少的操作次数。每次操作的可能性包括:
		
			\begin{itemize}
				\item[·] 删除元素。
				\item[·] 将元素加 1。
			\end{itemize}
			
	\end{itemize}
		

	\item[2.]递归思路:
	\begin{itemize}
	\item[·] 对于每个元素,我们有三个决策:
	
		\begin{itemize}
			\item[1.] 删除当前元素,即跳过该元素并继续递归计算后续元素的操作。
			\item[2.] 保留当前元素,并将其值加 1 后递归处理后续元素。
			\item[3.] 保留当前元素,不改变它的值,递归处理后续元素。
		\end{itemize}
		
	\item[·] 我们通过递归来遍历所有可能的操作路径,直到处理完所有元素。
	\item[·] 在每个递归步骤中,记录当前的序列和,并判断是否已经是 x 的倍数。
	\end{itemize}
	
	\item[3.]备忘录优化:
	\begin{itemize}
	\item[·] 为了避免重复计算相同的状态,使用一个二维数组 dp 来存储已经计算过的状态。
	\item[·] 状态由两个变量决定:
		\begin{itemize}
		\item[·] 当前处理的序列位置 i。
		\item[·] 当前序列的数字之和 \texttt{sum \% x}(我们只关心序列和对 x 的模,因为我们只需要知道和是否是 x 的倍数)。
		\end{itemize}
	\item[·] 如果在某个状态下已经计算过最少操作次数,直接从备忘录中读取结果,避免重复计算。
	\end{itemize}

	\item[4.]递归终止条件:
	
	\begin{itemize}
	\item[·] 当遍历完所有元素时(即下标 i 达到 \texttt{A.size()}),检查当前序列和是否是 x 的倍数。如果是,返回 0 表示不需要更多操作;否则返回 \texttt{INT\_MAX},表示该路径不可行。
	\end{itemize}
		
	\item[5.]递归步骤:
	\begin{itemize}
	\item[·] 我们在每个递归步骤中,计算三种操作的结果:
	
		\begin{itemize}
		\item[1.] 删除操作:调用递归处理后续元素,操作数加 1。
		\item[2.] 加 1 操作:当前元素加 1 后,调用递归处理后续元素,操作数加 1。
		\item[3.] 保留操作:直接保留当前元素,递归处理后续元素。
		\end{itemize}
		
	\item[·] 比较三者的操作次数,取最小值作为当前最优选择。
	\end{itemize}
\end{itemize}


\noindent\textbf{实现代码:}

\begin{lstlisting}
#include <iostream>
#include <vector>
#include <climits>

using namespace std;

// 递归函数,计算当前 idx 和 sum 下的最小操作数
int dfs(int i, int sum, int x, vector<int>& A, vector<vector<int>>& dp) {
	// 基本情况:如果已经检查到序列末尾
	if (i == A.size()) {
		return (sum % x == 0) ? 0 : 100001; // 判断当前和是否为 x 的倍数
	}
	
	// 检查备忘录,是否已经计算过该状态
	if (dp[i][sum % x] != -1) {
		return dp[i][sum % x]; // 返回已保存的结果
	}
	
	// 选择 1: 删除当前元素
	int del_op = dfs(i + 1, sum, x, A, dp);
	
	// 选择 2: 把当前元素加 1
	int add_op = dfs(i + 1, sum + A[i] + 1, x, A, dp);
	
	// 选择 3: 保留当前元素(不加一)
	int keep_op = dfs(i + 1, sum + A[i], x, A, dp);
	
	// 取三种选择的最小操作数
	int res = min(del_op + 1, min(add_op + 1, keep_op));
	
	// 保存当前状态到备忘录
	dp[i][sum % x] = res;
	return res;
}

int main() {
	int n, x;
	cin >> n >> x;
	
	// 输入序列
	vector<int> A(n);
	for (int i = 0; i < n; i++) {
		cin >> A[i];
	}
	
	// 初始化备忘录数组,-1 表示尚未计算
	vector<vector<int>> dp(n, vector<int>(x, -1));
	
	// 调用递归函数,从索引 0 开始,初始和为 0
	int result = dfs(0, 0, x, A, dp);
	
	// 输出结果,如果无法达成目标,则返回 -1
	if (result == 100001) {
		cout << -1 << endl;
	} else {
		cout << result << endl;
	}
	
	return 0;
}
\end{lstlisting}
	
	
\end{document}
