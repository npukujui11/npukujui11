\documentclass[a4paper,10pt]{article}

\usepackage{tabularx} % extra features for tabular environment
\usepackage{amsmath}  % improve math presentation
\usepackage{graphicx} % takes care of graphic including machinery
\usepackage[margin=1in,letterpaper]{geometry} % decreases margins
\usepackage{cite} % takes care of citations
\usepackage[final]{hyperref} % adds hyper links inside the generated pdf file
\usepackage{ctex}
\usepackage{titlesec}
%\usepackage{CJKutf8, CJK}
\usepackage{makecell}                 % 三线表-竖线
\usepackage{booktabs}                 % 三线表-短细横线
% \usepackage{natbib}
\usepackage{graphicx}				  % 表格单元格逆时针
\usepackage{multirow}				  % 合并单元格
\usepackage{array}
\usepackage{amssymb}				  % 勾
\usepackage{amsmath}
\usepackage{longtable}                % 导入 longtable 宏包,表格自动换行
\usepackage{caption}
\usepackage{subcaption}               % 设置子图
\usepackage{color}					  % 文本颜色包
\usepackage{xcolor}
\usepackage{bbm}					  % 输入指示函数
\usepackage{tablefootnote}			  % 表格注释
\usepackage{pythonhighlight}
\usepackage{fancyhdr}
\usepackage{lastpage}
\pagestyle{fancy}
\fancyhf{}
\fancyhead{}
\fancyfoot{}
\fancyhead[R]{\small Page \thepage\ of \pageref*{LastPage}}
\fancyhead[L]{\small Report}

\usepackage{listings}                 % 导入代码块
\usepackage{xcolor}
\lstset{
	numbers=left, 
	tabsize=1,
	columns=flexible, 
	numberstyle=  \small, 
	keywordstyle= \color{ blue!70},
	commentstyle= \color{red!50!green!50!blue!50}, 
	frame=shadowbox, % 阴影效果
	rulesepcolor= \color{ red!20!green!20!blue!20} ,
	escapeinside=``, % 英文分号中可写入中文
	xleftmargin=2em,
	xrightmargin=2em, 
	aboveskip=1em,
} 

\hypersetup{
	colorlinks=true,       % false: boxed links; true: colored links
	linkcolor=blue,        % color of internal links
	citecolor=blue,        % color of links to bibliography
	filecolor=magenta,     % color of file links
	urlcolor=blue         
}
%++++++++++++++++++++++++++++++++++++++++
\titleformat{\section}{\Large\bfseries\songti}{\thesection}{1em}{}
\titleformat{\subsection}{\large\bfseries\songti}{\thesubsection}{1em}{}
\titleformat{\subsubsection}{\normalsize\bfseries\songti}{\thesubsubsection}{1em}{}
\titleformat{\paragraph}{\small\bfseries\songti}{\paragraph}{1em}{}
\titleformat{\subparagraph}{\footnotesize\bfseries\songti}{\subparagraph}{1em}{}

\begin{document}
	
	
	\title{\songti \zihao{4}Unity VR 项目开发进度}
%	\author{\textrm{Ku Jui}}
	\date{\textrm{October 2023}}
	\maketitle
	
	\renewcommand{\figurename}{Figure} % 可以重新定义abstract,因为ctex会覆盖thebibliography
	% 	\begin{abstract}
		%		In this experiment we studied a very important physical effect by measuring the
		%		dependence of a quantity $V$ of the quantity $X$ for two different sample
		%		temperatures.  Our experimental measurements confirmed the quadratic dependence
		%		$V = kX^2$ predicted by Someone's first law. The value of the mystery parameter
		%		$k = 15.4\pm 0.5$~s was extracted from the fit. This value is
		%		not consistent with the theoretically predicted $k_{theory}=17.34$~s. We attribute %this
		%		discrepancy to low efficiency of our $V$-detector.
		%	\end{abstract}
	\renewcommand{\contentsname}{Contents}
	\renewcommand{\tablename}{Table}
	\tableofcontents  % 自动生成目录
		
	\section{开发进度}
	
		开发项目演示视频见\footnote{https://github.com/npukujui11/npukujui11/tree/Unity-VR/LaTeX/assignment/Unity-VR/Progress-report1/video},附图均为游戏实机画面。

		\subsection{游戏界面菜单}	
		
		进入游戏界面(见Fig. \ref{fig: Main menu}),通过手柄光标选择天气和船只,通过 $A$ 键确认选择,$B$ 键关闭手柄光标。
		
		\begin{figure}[htbp] 
			% read manual to see what [ht] means and for other possible options
			\centering 
			% \includegraphics[width=0.8\columnwidth]{GLADNet}
			
			\begin{subfigure}{0.23\textwidth}
				\includegraphics[width=\linewidth]{picture/Main menu-IP}
				\captionsetup{font=scriptsize}
				\caption{Main menu}
				\label{fig: Main menu-IP}
			\end{subfigure}
			\begin{subfigure}{0.23\textwidth}
				\includegraphics[width=\linewidth]{picture/Main menu-IP-1}
				\captionsetup{font=scriptsize}
				\caption{Option}
				\label{fig: Main menu-IP-1}
			\end{subfigure}
			\begin{subfigure}{0.23\textwidth}
				\includegraphics[width=\linewidth]{picture/Main menu-IP-2}
				\captionsetup{font=scriptsize}
				\caption{Cursor}
				\label{fig: Main menu-IP-2}
			\end{subfigure}
			\begin{subfigure}{0.23\textwidth}
				\includegraphics[width=\linewidth]{picture/Main menu-IP-3}
				\captionsetup{font=scriptsize}
				\caption{Boat}
				\label{fig: Main menu-IP-3}
			\end{subfigure}
			
			\captionsetup{font=scriptsize}
			\caption{
				\label{fig: Main menu}	
				VR游戏主界面示意图				
			}
		\end{figure}
	
		\subsection{场景开发进度}
		
		我们已经开发了三种不同的天气场景,分别是雾天 (Foggy) 、下雨天 (Rainy)和冰雹天气 (Hail)。这些场景的视觉效果分别展示在Fig .\ref{fig: Third-person perspective another-IP}, Fig .\ref{fig: Third-person perspective another rain-IP}, Fig .\ref{fig: Third-person perspective another hail-IP}。
		
		\begin{figure}[htbp] 
			% read manual to see what [ht] means and for other possible options
			\centering 
			% \includegraphics[width=0.8\columnwidth]{GLADNet}
			
			
			\begin{subfigure}{0.3\textwidth}
				\includegraphics[width=\linewidth]{picture/Third-person perspective another hail-IP}
				\captionsetup{font=scriptsize}
				\caption{Hail}
				\label{fig: Third-person perspective another hail-IP}	
			\end{subfigure}
			\captionsetup{font=scriptsize}
			\caption{
				\label{fig: Scene}	
				第三人称视角下的不同天气场景					
			}
		\end{figure}
	
		\subsection{视角开发进度}
		
		玩家通过按下手柄摇杆按钮控制船只前进和后退,以及转向行驶\footnote{https://github.com/npukujui11/npukujui11/tree/Unity-VR/LaTeX/assignment/Unity-VR/Progress-report1/video}。
		
		玩家开船可以选择不同的船只视角,我们目前已经开发右置手柄适配,通过 \textbf{RT} 按键,可以切换视角,如 Fig. \ref{fig: First-person perspective Hail-IP} 和 Fig .\ref{fig: Third-person perspective another hail-IP1} 所示。
		
		\begin{figure}[htbp] 
			% read manual to see what [ht] means and for other possible options
			\centering 
			% \includegraphics[width=0.8\columnwidth]{GLADNet}
			
			\begin{subfigure}{0.45\textwidth}
				\includegraphics[width=\linewidth]{picture/First-person perspective Hail-IP}
				\captionsetup{font=scriptsize}
				\caption{First-person}
				\label{fig: First-person perspective Hail-IP}
			\end{subfigure}
			\begin{subfigure}{0.45\textwidth}
				\includegraphics[width=\linewidth]{picture/Third-person perspective another hail-IP1}
				\captionsetup{font=scriptsize}
				\caption{Third-person}
				\label{fig: Third-person perspective another hail-IP1}
			\end{subfigure}
			\captionsetup{font=scriptsize}
			\caption{
				\label{fig: View}	
				同一天气下的不同视角					
			}
		\end{figure}
		
		\subsection{船只模型}
		
		目前游戏我们已针对两种船只模型做了模型适配,见 Fig. \ref{fig: Main menu-IP-2} 和 Fig. \ref{fig: Main menu-IP-3}
		
		\subsection{未来开发计划}
		
		基于构建 VR 沉浸式体验装置的需求,计划通过添加游戏控制器震动反馈作为对气缸控制的前期预研,以期解决目前气缸无序运动的问题。同时计划在 VR 游戏中构建更合理的海洋水系统,以期构建更多样化的天气场景。		
		
		\subsection{实机画面}
		
		\begin{figure}[htb]
			% read manual to see what [ht] means and for other possible options
			\centering				
			\includegraphics[width=0.9\columnwidth]{picture/Third-person-perspective-Storm}
			%\captionsetup{font=scriptsize}
			\caption{
				\label{fig: Third-person-perspective-Storm} 
				暴风雨天气
			}	
		\end{figure}
		
		\begin{figure}[htbp]
			% read manual to see what [ht] means and for other possible options
			\centering				
			\includegraphics[width=0.9\columnwidth]{picture/Third-person perspective-IP}
			%\captionsetup{font=scriptsize}
			\caption{
				\label{fig: Third-person perspective-IP} 
				雾天
			}	
		\end{figure}
		
		\begin{figure}[htbp]
			% read manual to see what [ht] means and for other possible options
			\centering				
			\includegraphics[width=0.9\columnwidth]{picture/First-person perspective storm-IP}
			%\captionsetup{font=scriptsize}
			\caption{
				\label{fig: First-person perspective storm-IP} 
				第一人称视角
			}	
		\end{figure}
		

	%	\section{Analysis}
	
	%	In this section you will need to show your experimental results. Use tables and
	%	graphs when it is possible. Table~\ref{tbl:bins} is an example.
	
	%	\begin{table}[ht]
		%		\begin{center}
			%			\caption{Every table needs a caption.}
			%			\label{tbl:bins} % spaces are big no-no withing labels
			%			\begin{tabular}{|ccc|} 
				%				\hline
				%				\multicolumn{1}{|c}{$x$ (m)} & \multicolumn{1}{c|}{$V$ (V)} & \multicolumn{1}{c|}{$V$ (V)} \\
				%				\hline
				%				0.0044151 &   0.0030871 &   0.0030871\\
				%				0.0021633 &   0.0021343 &   0.0030871\\
				%				0.0003600 &   0.0018642 &   0.0030871\\
				%				0.0023831 &   0.0013287 &   0.0030871\\
				%				\hline
				%			\end{tabular}
			%		\end{center}
		%	\end{table}
	%	
	%	Analysis of equation~\ref{eq:aperp} shows ...
	%	
	%	Note: this section can be integrated with the previous one as long as you
	%	address the issue. Here explain how you determine uncertainties for different
	%	measured values. Suppose that in the experiment you make a series of
	%	measurements of a resistance of the wire $R$ for different applied voltages
	%	$V$, then you calculate the temperature from the resistance using a known
	%	equation and make a plot  temperature vs. voltage squared. Again suppose that
	%	this dependence is expected to be linear~\cite{Cyr}, and the proportionality coefficient
	%	is extracted from the graph. Then what you need to explain is that for the
	%	resistance and the voltage the uncertainties are instrumental (since each
	%	measurements in done only once), and they are $\dots$. Then give an equation
	%	for calculating the uncertainty of the temperature from the resistance
	%	uncertainty. Finally explain how the uncertainty of the slop of the graph was
	%	found (computer fitting, graphical method, \emph{etc}.)
	%	
	%	If in the process of data analysis you found any noticeable systematic
	%	error(s), you have to explain them in this section of the report.
	%	
	%	It is also recommended to plot the data graphically to efficiently illustrate
	%	any points of discussion. For example, it is easy to conclude that the
	%	experiment and theory match each other rather well if you look at
	%	Fig.~\ref{fig:samplesetup} and Fig.~\ref{fig:exp_plots}.
	%	
	%	\begin{figure}[ht] 
		%		\centering
		%		\includegraphics[width=0.5\columnwidth]{sr_squeezing_vs_detuning}
		%		
		%		% some figures do not need to be too wide
		%		\caption{
			%			\label{fig:exp_plots}  
			%			Every plot must have axes labeled.
			%		}
		%	\end{figure}
	
	
	%	\section{Conclusions}
	%	Here you briefly summarize your findings.
	
	%++++++++++++++++++++++++++++++++++++++++
	% References section will be created automatically 
	% with inclusion of "thebibliography" environment
	% as it shown below. See text starting with line
	% \begin{thebibliography}{99}
		% Note: with this approach it is YOUR responsibility to put them in order
		% of appearance.
		
%		\renewcommand{\refname}{References}
		
		
		%	\begin{thebibliography}{00}
			
			%		\bibitem{b1}\label{cite:b1}
			%		W. Wang, C. Wei, W. Yang and J. Liu, "GLADNet: Low-Light Enhancement Network with Global Awareness," 2018 13th IEEE International Conference on Automatic Face \& Gesture Recognition (FG 2018), Xi'an, China, 2018, pp. 751-755, DOI: 10.1109/FG.2018.00118.
			
			%		\bibitem{b2}\label{cite:b2}
			%		A.\ Mahajan, K.\ Somaraj and M. Sameer, "Adopting Artificial Intelligence Powered ConvNet To Detect Epileptic Seizures," 2020 IEEE-EMBS Conference on Biomedical Engineering and Sciences (IECBES), Langkawi Island, Malaysia, 2021, pp. 427-432, DOI: 10.1109/IECBES48179.2021.9398832.
			
			%		\bibitem{Cyr}
			%		N.\ Cyr, M.\ T$\hat{e}$tu, and M.\ Breton,
			% "All-optical microwave frequency standard: a proposal,"
			%		IEEE Trans.\ Instrum.\ Meas.\ \textbf{42}, 640 (1993).
			
			
			
			%	\end{thebibliography}
		
%		\bibliographystyle{unsrt}
%		\bibliography{reference}
		
		
	\end{document}
