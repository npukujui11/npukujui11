\documentclass[a4paper]{ctexart}

\usepackage{tabularx} % extra features for tabular environment
\usepackage{amsmath}  % improve math presentation
\usepackage{graphicx} % takes care of graphic including machinery
\usepackage[margin=1in,letterpaper]{geometry} % decreases margins
\usepackage{cite} % takes care of citations
\usepackage[final]{hyperref} % adds hyper links inside the generated pdf file
\usepackage{ctex}
\usepackage{titlesec}
%\usepackage{CJKutf8, CJK}
\usepackage{makecell}                 % 三线表-竖线
\usepackage{booktabs}                 % 三线表-短细横线
% \usepackage{natbib}
\usepackage{graphicx}				  % 表格单元格逆时针
\usepackage{multirow}				  % 合并单元格
\usepackage{array}
\usepackage{amssymb}				  % 勾
\usepackage{amsmath}
\usepackage{longtable}                % 导入 longtable 宏包,表格自动换行
\usepackage{caption}
\usepackage{subcaption}               % 设置子图
\usepackage{caption}
%\usepackage{subfigure}
\usepackage{diagbox}
\usepackage{color}					  % 文本颜色包
\usepackage{colortbl}
\usepackage{xcolor}
\usepackage{bbm}					  % 输入指示函数
\usepackage{tablefootnote}			  % 表格注释
\usepackage{pythonhighlight}
%\usepackage{fancyhdr}
\usepackage{lastpage}
\usepackage{tocloft}
\usepackage{authblk}
\usepackage{setspace}
\usepackage{float}

\usepackage[section]{placeins}

% 设置页面边距
\geometry{a4paper, top=1.7cm, bottom=1.6cm, left=1.6cm, right=1.6cm}

%\pagestyle{fancy}
%\fancyhf{}
%\fancyhead{}
%\fancyfoot{}
%\fancyhead[R]{\small Page \thepage\ of \pageref*{LastPage}}
%\fancyhead[L]{\zihao{-5} \songti 开题报告}

\usepackage{listings}                 % 导入代码块
\usepackage{xcolor}
\lstset{
	numbers=left, 
	tabsize=1,
	columns=flexible, 
	numberstyle=  \small, 
	keywordstyle= \color{ blue!70},
	commentstyle= \color{red!50!green!50!blue!50}, 
	frame=shadowbox, % 阴影效果
	rulesepcolor= \color{ red!20!green!20!blue!20} ,
	escapeinside=``, % 英文分号中可写入中文
	xleftmargin=2em,
	xrightmargin=2em, 
	aboveskip=1em,
} 

\hypersetup{
	colorlinks=true,       % false: boxed links; true: colored links
	linkcolor=blue,        % color of internal links
	citecolor=blue,        % color of links to bibliography
	filecolor=magenta,     % color of file links
	urlcolor=blue         
}
%++++++++++++++++++++++++++++++++++++++++
\titleformat{\section}{\Large\bfseries}{\thesection}{1em}{}
\titleformat{\subsection}{\large\bfseries}{\thesubsection}{1em}{}
\titleformat{\subsubsection}{\normalsize\bfseries}{\thesubsubsection}{1em}{}
\titleformat{\paragraph}[runin]{\normalsize\bfseries}{\paragraph}{1em}{}
\titleformat{\subparagraph}[runin]{\normalsize\bfseries}{\subparagraph}{1em}{}

\begin{document}

(2024年新课标II)Given the astonishing potential of AI to transform our lives, we all need to take action to deal with our AI-powered future, and this is where AI by Design: A Plan for Living with Artificial Intelligence comes in. This absorbing new book by Catriona Campbell is a practical roadmap addressing the challenges posed by the forthcoming AI revolution(变革). 

\underline{In the wrong hands}, such a book could prove as complicated to process as the computer code(代码)that powers AI but, thankfully, Campbell has more than two decades' professional experience translating the heady into the understandable. She writes from the practical angle of a business person rather than as an academic, making for a guide which is highly accessible and informative and which, by the close, will make you feel almost as smart as AI. 

As we soon come to learn from \textit{AI by Design}, AI is already super-smart and will become more capable, moving from the current generation of "narrow-AI" to Artificial General Intelligence. From there, Campbell says, will come Artificial Dominant Intelligence. This is why Campbell has set out to raise awareness of AI and its future now-several decades before these developments are expected to take place. She says it is essential that we keep control of artificial intelligence, or risk being sidelined and perhaps even worse. 

Campbell's point is to wake up those responsible for AI-the technology companies and world leaders-so they are on the same page as all the experts currently developing it. She explains we are at a "tipping point" in history and must act now to prevent an extinction-level event for humanity. We need to consider how we want our future with Al to pan out. Such structured thinking, followed by global regulation, will enable us to achieve greatness rather than our downfall. 

AI will affect us all, and if you only read one book on the subject, this is it. 


32. What does the phrase "In the wrong hands" in paragraph 2 probably mean? 

A. If read by someone poorly educated.    
     
B. If reviewed by someone ill-intentioned.

C. If written by someone less competent.       
  
D. If translated by someone unacademic. 

33. What is a feature of \textit{AI by Design} according to the text? 

A. It is packed with complex codes.            

B. It adopts a down-to-earth writing style.

C. It provides step-by-step instructions.     
    
D. It is intended for AI professionals. 

34. What does Campbell urge people to do regarding AI development? 

A. Observe existing regulations on it.

B. Reconsider expert opinions about it.  

C. Make joint efforts to keep it under control. 

D. Learn from prior experience to slow it down. 

35. What is the author's purpose in writing the text? 

A. To recommend a book on AI.         

B. To give a brief account of AI history. 

C. To clarify the definition of AI.        

D. To honor an outstanding AI expert.


\newpage

(2024新课标II)When I decided to buy a house in Europe ten years ago, I didn't think too long. I liked traveling in France, but when it came to picking my favorite spot to  \underline{\hspace{0.5cm}41 \hspace{0.5cm}}, Italy was the clear winner. During my first visit to Italy, I \underline{\hspace{0.5cm}42 \hspace{0.5cm}} to ask for directions or order in a restaurant. But every time I tried to \underline{\hspace{0.5cm}43 \hspace{0.5cm}} a sentence of Italian together, the locals smiled at me and \underline{\hspace{0.5cm}44 \hspace{0.5cm}} my language skills. That encouragement helped me to get through the language \underline{\hspace{0.5cm}45\hspace{0.5cm}}. After I made Italy my permanent home, I discovered how \underline{\hspace{0.5cm}46 \hspace{0.5cm}} Italians are. Neighbors will bring me freshly made cheese and will come to my door to \underline{\hspace{0.5cm}47 \hspace{0.5cm}} me to close the window in my car when rain is coming. It's these small \underline{\hspace{0.5cm}48 \hspace{0.5cm}} of kindness that make a new country feel like home. 
As a foodie, the way to my heart is through my stomach, and nowhere fuels my \underline{\hspace{0.5cm}49\hspace{0.5cm}} quite like Italy. Each town has its own traditional \underline{\hspace{0.5cm}50\hspace{0.5cm}}, and every family keeps a recipe passed from one generation to another. Families \underline{\hspace{0.5cm}51\hspace{0.5cm}} for big meals on Sundays, birthdays, and whatever other excuses they can \underline{\hspace{0.5cm}52 \hspace{0.5cm}}. These meals are always  \underline{\hspace{0.5cm}53 \hspace{0.5cm}}by laughter and joy. Whatever\underline{\hspace{0.5cm}54 \hspace{0.5cm}}life in Italy might have, the problems are \underline{\hspace{0.5cm}55 \hspace{0.5cm}} once you sit down to a big meal with friends and family. 

\begin{tabular}{>{\raggedright\arraybackslash}p{0.2cm} p{3cm} p{3cm} p{3cm} p{3cm}}
	41. & A.study & B.rent & C. visit & D. settle \\[0.3cm]
	42. & A.planned & B.struggled & C. refused & D. happened \\[0.3cm]
	43. & A.string & B.hang & C. mix & D. match \\[0.3cm]
	44. & A.improved & B.assessed & C. admired & D. praised \\[0.3cm]
	45. & A.course & B.barrier & C. area & D. test \\[0.3cm]
	46. & A.open-minded & B.strong-willed & C. warm-hearted & D. well-informed \\[0.3cm]
	47. & A.remind & B.allow & C. persuade & D. order \\[0.3cm]
	48. & A.tricks & B.promises & C. acts & D. duties \\[0.3cm]
	49. & A.ambition & B.success & C. appetite & D. growth \\[0.3cm]
	50. & A.costume & B.dish & C. symbol & D. tale \\[0.3cm]
	51. & A.gather & B.cheer & C. leave & D. wait \\[0.3cm]
	52. & A.put up with & B.stand up for & C. come up with & D. make up for \\[0.3cm]
	53. & A.signaled & B.confirmed & C. represented & D. accompanied \\[0.3cm]
	54. & A.disadvantages & B.meanings & C. surprises & D. opportunities \\[0.3cm]
	55. & A.created & B.forgotten & C. understood & D. identified \\[0.3cm]
\end{tabular}

\newpage

41.D.settle - 上下文表明,说话者决定让意大利成为他们永久的家,所以“settle”是最合适的选择。

42.B.struggled - 说话者提到他在语言上有困难,表明他在沟通上有困难.

43.A.string -“串在一起”的意思是连接或形成一个系列,这符合造句的上下文。

44.A.improved - 当地人的鼓励帮助演讲人提高了他们的语言技能.

45.B.barrier - 演讲者在试图用意大利语交流时遇到了语言障碍。

46.C.warm-hearted - 邻居的行为,比如带来新鲜的奶酪和帮助打开车窗,都表明了他的热心。

47.A.remind -邻居们提醒演讲者在下雨的时候关上车窗。

48.C.acts - 邻居们的小小善举使这个新国家有了家的感觉。

49.C.appetite - 作为一个吃货,说话者对食物的热爱用“appetite”来表示。

50.B.dish - 每个城镇都有自己的传统菜肴,适合食物和烹饪的背景。

51.A.gather -一家人聚在一起吃大餐,表明这是一种集体活动。

52.C.come up with - 家人会找各种借口聚在一起吃饭。

53.D.accompanied - 吃饭总是伴随着笑声和欢乐。

54.D.opportunities - 上下文表明,在意大利生活有很多机会,但在享受美食时却被遗忘了。

55.B.forgotten -一旦坐下来吃饭,问题就被忘记了。
	
\end{document}
