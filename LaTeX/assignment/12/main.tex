\documentclass[a4paper, 10pt]{article}

\usepackage{tabularx} % extra features for tabular environment
\usepackage{amsmath}  % improve math presentation
\usepackage{graphicx} % takes care of graphic including machinery
\usepackage[margin=1in,letterpaper]{geometry} % decreases margins
\usepackage{cite} % takes care of citations
\usepackage[final]{hyperref} % adds hyper links inside the generated pdf file
\usepackage{ctex}
\usepackage{titlesec}
%\usepackage{CJKutf8, CJK}
\usepackage{makecell}                 % 三线表-竖线
\usepackage{booktabs}                 % 三线表-短细横线
% \usepackage{natbib}
\usepackage{graphicx}				  % 表格单元格逆时针
\usepackage{multirow}				  % 合并单元格
\usepackage{array}
\usepackage{amssymb}				  % 勾
\usepackage{amsmath}
\usepackage{longtable}                % 导入 longtable 宏包,表格自动换行
\usepackage{caption}
\usepackage{subcaption}               % 设置子图
\usepackage{color}					  % 文本颜色包
\usepackage{xcolor}
\usepackage{bbm}					  % 输入指示函数
\usepackage{tablefootnote}			  % 表格注释
\usepackage{pythonhighlight}
\usepackage{fancyhdr}
\usepackage{lastpage}
\pagestyle{fancy}
\fancyhf{}
\fancyhead{}
\fancyfoot{}
\fancyhead[R]{\small Page \thepage\ of \pageref*{LastPage}}
\fancyhead[L]{\small Report}

\usepackage{listings}                 % 导入代码块
\usepackage{xcolor}
\lstset{
	numbers=left, 
	tabsize=1,
	columns=flexible, 
	numberstyle=  \small, 
	keywordstyle= \color{ blue!70},
	commentstyle= \color{red!50!green!50!blue!50}, 
	frame=shadowbox, % 阴影效果
	rulesepcolor= \color{ red!20!green!20!blue!20} ,
	escapeinside=``, % 英文分号中可写入中文
	xleftmargin=2em,
	xrightmargin=2em, 
	aboveskip=1em,
} 

\hypersetup{
	colorlinks=true,       % false: boxed links; true: colored links
	linkcolor=blue,        % color of internal links
	citecolor=blue,        % color of links to bibliography
	filecolor=magenta,     % color of file links
	urlcolor=blue         
}
%++++++++++++++++++++++++++++++++++++++++
\titleformat{\section}{\Large\bfseries\songti}{\thesection}{1em}{}
\titleformat{\subsection}{\large\bfseries\songti}{\thesubsection}{1em}{}
\titleformat{\subsubsection}{\normalsize\bfseries\songti}{\thesubsubsection}{1em}{}
\titleformat{\paragraph}{\small\bfseries\songti}{\paragraph}{1em}{}
\titleformat{\subparagraph}{\footnotesize\bfseries\songti}{\subparagraph}{1em}{}

\begin{document}
	
	
	\title{\songti \zihao{4}工作简报}
	\author{\textrm{Ku Jui}}
	\date{\textrm{December 2023}}
	\maketitle
	
	\renewcommand{\figurename}{Figure} % 可以重新定义abstract,因为ctex会覆盖thebibliography
	% 	\begin{abstract}
		%		In this experiment we studied a very important physical effect by measuring the
		%		dependence of a quantity $V$ of the quantity $X$ for two different sample
		%		temperatures.  Our experimental measurements confirmed the quadratic dependence
		%		$V = kX^2$ predicted by Someone's first law. The value of the mystery parameter
		%		$k = 15.4\pm 0.5$~s was extracted from the fit. This value is
		%		not consistent with the theoretically predicted $k_{theory}=17.34$~s. We attribute %this
		%		discrepancy to low efficiency of our $V$-detector.
		%	\end{abstract}
	\renewcommand{\contentsname}{目录}
	\renewcommand{\tablename}{Table}
	\newpage
	
	\part*{ \zihao{4} 完成 CDP-VR Tester (Windows 端) 代码开发}
	\section{主要成果}
	
	经过对功能业务逻辑的沟通和磋商,目前已经完成了 CDP-VR Tester (Windows 端) 全部的代码部分,主要完了以下几个功能。
	
	\subsection{蓝牙连接功能}
	
	\begin{itemize}
		\item [+] connectBluetooth(): 连接蓝牙设备。	
			\begin{itemize}
				\item [-] FindBtAddress(): 查找特定名称的蓝牙设备地址。
				\item [-] DisconnectBluetooth(): 断开蓝牙设备连接。
			\end{itemize}
		\item [+] BluetoothReadThread: 定义线程接收蓝牙设备数据。
			\begin{itemize}
				\item [-] ProcessReadData(): 处理接收到的 VR 眼镜数据。
			\end{itemize}
	\end{itemize}
	
	\subsection{蓝牙数据传输}
	
	\begin{itemize}
		\item [+] BluetoothSendData(): 定义蓝牙数据缓冲区和发送逻辑。
			\begin{itemize}
				\item [-] SendData(): 负责数据发送。
			\end{itemize}
	\end{itemize}
	
	\subsection{蓝牙缓冲区定义}
	
	\begin{itemize}
		\item [$\bullet$] 长度:9个字节。
		\item [$\bullet$] 字节定义:
			
			第1个字节:存放 CurMinorVersion(次版本号)。
			
			第2个字节:存放 CurMajorVersion(主版本号)。
			
			第3个字节:传递 Seethrough 指令。
			
			第4个字节:传递 Darkness 指令。
			
			第5个字节:传递 Freeze Pitch 指令。
			
			第6个字节:传递 Pitch 值。
			
			第7个字节:传递场景值。
			
			第8和9个字节:保留。
	\end{itemize}
	
	\subsection{当前工作}
	
	VR Goggle端正在进行蓝牙缓冲区读取功能的代码修改,以确保能按照指定的字节定义来读取接收到的蓝牙数据包。
	
	\subsection{下一步计划}
	
	第一轮测试工作计划于下周一和周二进行。
	
	\part*{\zihao{4} 开题报告文档和PPT制作}
	
	\section{主要工作}
	
	\begin{itemize}
		\item[(1)] 使用 \LaTeX 编写开题报告文档和PPT。
		
		\item[(2)] 按照信息院官网发布的要求,报告主要章节包括:
		
			\begin{itemize}
				\item [1] 研究介绍		
					\begin{itemize}
						\item [\checkmark 1.1] 研究意义
						\item [\checkmark 1.2] 研究背景和现状
						\begin{itemize}
							\item [\checkmark 1.2.1] 传统方法
							\item [\checkmark 1.2.2] 基于深度学习的低照度图像增强方法
							\item [\checkmark 1.2.3] 研究背景
						\end{itemize}						
					\end{itemize}				
				\item [2] 研究内容			
					\begin{itemize}
						\item [\checkmark 2.1] 采用的方法
						\item [\checkmark 2.2] 数据集
						\item [\checkmark 2.3] 文献调研与支撑(标题待定)
					\end{itemize}			
				\item [3] 具体工作内容				
					\begin{itemize}
						\item [3.1] 已完成工作
						\item [3.2] 下一步工作
					\end{itemize}
			\end{itemize}
		
		\item[(3)] 目前已完成第1章和第2章的文档编写,PPT进度与文档一致。
		
			\item[\checkmark] 正在考虑是否删除2.2章节(数据集)。
		
		\item[(4)] 开题报告文档预计页数约30页,引用文献约200条。
	\end{itemize}
	
	\section{下一步计划}
	
		\begin{itemize}
			\item [(1)] 完成文档:
				
				\item [\checkmark] 优先完成文档编写(预计下周末完成)。
				
				\item [\checkmark] 定义并统一术语表示,如“Low-light Image”统一称为“低光图像”。
				
				\item [\checkmark] 整理文档格式并检查内容完整性。
				
			\item [(2)] 格式与规范:
			
				\item [\checkmark] 学院暂未明确文档和PPT的格式求,可能需要后续调整。
		\end{itemize}
	
%	\renewcommand{\refname}{References}
%	\bibliographystyle{unsrt}
%	\bibliography{reference}
	
	
\end{document}
